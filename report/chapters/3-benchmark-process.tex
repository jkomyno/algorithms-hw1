\section{Benchmark}
\label{cap:benchmark-process}

Abbiamo deciso di rendere il processo di misurazione del tempo di esecuzione dei nostri algoritmi esterno al codice dei programmi sviluppati.
Il nostro benchmark tiene quindi conto di tutto, ovvero:

\begin{itemize}
    \item Tempo necessario a leggere il file di input in un vettore temporaneo;
    \item Tempo per trasformare il vettore temporaneo in una lista di adiacenza;
    \item Tempo per eseguire l'algoritmo vero e proprio per il calcolo dell'\textbf{MST};
    \item Tempo per scorrere l'\textbf{MST} ritornato dall'algoritmo per calcolarne il peso totale.
\end{itemize}

\noindent Lo script Powershell \textit{benchmark.ps1}, che richiama a sua volta lo script \textit{time.ps1}, si occupa di misurare il tempo richiesto a ogni programma sviluppato per questo homework per calcolare il peso del Minimum Spanning Tree di ogni dataset di input. Al termine dell'esecuzione, esso genera un file CSV per ogni programma di cui è stato fatto il benchmark. Il nome di ogni file CSV generato contiene il nominativo del programma misurato e il timestamp in cui è stato creato. \\

\noindent Le colonne dei file CSV sopracitati sono le seguenti:

\begin{itemize}
    \item \textbf{ms}: tempo in millisecondi per eseguire il programma su un singolo file di input;
    \item \textbf{output}: risultato del programma, ovvero peso dell'\textbf{MST} del grafo letto in input;
    \item \textbf{n}: numero di nodi del grafo letto;
    \item \textbf{m}: numero di archi del grafo letto;
    \item \textbf{filename}: nome del file di input letto.
\end{itemize}

\noindent Per rendere i risultati del benchmark quanto più stabili e affidabili possibile, abbiamo preso le seguenti precauzioni:

\begin{itemize}
    \item Abbiamo usato sempre lo stesso computer per misurare il tempo di esecuzione dei programmi implementati;
    \item Abbiamo chiuso tutti i programmi in foreground e disabilitato quanti più servizi possibile in background;
    \item Abbiamo disabilitato la connessione Internet del computer scelto
    \item Abbiamo fatto più misurazioni, e abbiamo tenuto la media delle misurazioni effettuate.
\end{itemize}
