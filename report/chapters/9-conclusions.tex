\section{Conclusioni}
\label{cap:conclusions}

In questa relazione abbiamo ampiamente gli algoritmi, le scelte implementative, i tempi di esecuzioni e i ``tricks'' usati per rendere i programmi più efficienti. \\

\noindent sezione \ref{cap:performance-analysis} abbiamo risposto alle 2 principali domande dell'homework, mentre in \ref{cap:benchmark-process} abbiamo descritto il processo di benchmark adottato, pensato per essere quanto più affidabile e stabile.
Nelle sezioni \ref{cap:code-structure}, \ref{cap:implementation-choices} e \ref{cap:algorithms} abbiamo invece discusso i dettagli tecnici, quali struttura, scelte implementative e codice degli algoritmi. \\

\noindent  La sezione \ref{cap:tests} descrive il processo di testing automatico che abbiamo adottato, che c'ha permesso di rilasciare versioni migliori degli algoritmi in più iterazioni, accorgendoci immediatamente di eventuali regressioni. La sezione \ref{cap:extensions-and-originalities} descrive infine le estensioni esplorate per soddisfare la nostra curiosità e arricchire il nostro bagaglio accademico, garantendo comunque la correttezza dei risultati. \\

\noindent L'algoritmo a cui abbiamo dedicato più tempo ed energie è, paradossalmente, KruskalNaive. Sapevamo sin dall'inizio che sarebbe stato l'algoritmo con i tempi di esecuzione più lenti, ma ci siamo ugualmente adoperati per cercare di ottimizzare l'implementazione di DFS per l'individuazione di cicli nel grafo, migliorando le performance di circa il 30\% per i grafi più complessi del dataset.

\noindent Questo progetto ci ha permesso di sperimentare più approcci, implementare e debuggare strutture dati non banali ``from scratch'' e valutare con attenzione che complessità temporali e che operazioni debba offrire una struttura dati per rendere efficienti algoritmi che la sfruttano. Questo homework ci ha permesso inoltre di migliorare la nostra comprensione del linguaggio di programmazione scelto e di come funziona la collaborazione da remoto di un piccolo team di persone. \\

\noindent Questo progetto è disponibile anche come repository pubblica su Github:

\begin{center}
\href{https://github.com/jkomyno/algorithms-hw1}{github.com/jkomyno/algorithms-hw1}
\end{center}
