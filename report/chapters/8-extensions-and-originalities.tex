\section{Estensioni e originalità}
\label{cap:extensions-and-originalities}

Oltre alle tre implementazioni richieste dalla consegna dell'homework, abbiamo deciso di esplorare qualche altra estensione degli algoritmi per il calcolo del Minimum Spanning Tree visti a lezione.

\subsection{Prim con K-ary Heap}

Dal punto di vista teorico, le Fibonacci Heap hanno una complessità temporale migliore delle K-ary Heap.
Tuttavia, in alcuni casi, le K-ary Heap hanno una performance migliore perché sfruttano meglio la \textit{cache locality}, e perché la Fibonacci Heap hanno un coefficiente di complessità nascosto piuttosto elevato. Abbiamo deciso di implementare la struttura dati K-ary heap per vedere se siano utili a migliorare le performance temporali dell'algoritmo di Prim.
Un altro motivo per cui abbiamo deciso di implementare le K-ary anziché le Fibonacci Heap, è che queste ultime sono molto più complesse da implementare. \\

\noindent Quando le K-ary heap sono usate per implementare code di priorità, l'operazione di aggiornamento del valore della chiave è più veloce rispetto ad una Binary Heap (infatti si ha \complexityLogN{} per le Binary Heap contro \complexityLogkN{} per le k-ary Heap).
L'operazione di rimozione dell'elemento con minore chiave, tuttavia, aumenta a \complexityKLogkN{} rispetto a \complexityLogN{} delle Binary Heap. Ma visto che nell'algoritmo di Prim le operazioni di cambio valore delle chiavi sono più comuni delle operazioni di estrazioni del minimo elemento, le K-ary Heap sono comunque, in linea teorica, più efficienti delle Binary Heap per quell'algoritmo. \\
% ...

\noindent Le figure \ref{fig:Prim2vsPrim4-500}, \ref{fig:Prim2vsPrim4-2k-4k}
e \ref{fig:Prim2vsPrim4-80k-100k} mostrano una comparativa tra le performance
dell'algoritmo con una binary heap e un k-ary heap dove $k = 4$.
Da \ref{fig:Prim2vsPrim4-500} e \ref{fig:Prim2vsPrim4-2k-4k} si nota
un lieve miglioramento delle performance utilizzando K-ary Heap,
anche se questo comporta più oscillazione dei tempi rispetto a input
differenti. La figura \ref{fig:Prim2vsPrim4-80k-100k} invece mostra
una controtendenza su taglie dell'input intorno agli 80k nodi, dove
PrimKHeap non riesce a migliorare le performance rispetto a
PrimBinaryHeap. Dal grafico però si nota un'oscillazione che copre
non ci permette di asserire con certezza quale dei due sia il migliore.

\begin{figure}[H]
    \centering
    \includegraphics[width=1.0\textwidth]{./images/PrimBinaryHeap_vs_PrimKHeap_(500_nodes).png}
	\caption{Andamento di PrimBinaryHeap e PrimKaryHeap con taglia dell'input da 0 a 500 nodi.}
    \label{fig:Prim2vsPrim4-500}
\end{figure}

\begin{figure}[H]
    \centering
    \includegraphics[width=1.0\textwidth]{./images/PrimBinaryHeap_vs_PrimKHeap_(nodes_between_2k_and_4k).png}
	\caption{Andamento di PrimBinaryHeap e PrimKaryHeap con taglia dell'input da 2k a 4k nodi.}
    \label{fig:Prim2vsPrim4-2k-4k}
\end{figure}

\begin{figure}[H]
    \centering
    \includegraphics[width=1.0\textwidth]{./images/PrimBinaryHeap_vs_PrimKHeap_(nodes_between_80k_and_100k).png}
	\caption{Andamento di PrimBinaryHeap e PrimKaryHeap con taglia dell'input da 80k a 100k nodi.}
    \label{fig:Prim2vsPrim4-80k-100k}
\end{figure}

\subsection{Kruskal con Disjoint-Set e path-compression}

In generale, KruskalUnionFind si comporta leggermente meglio di
KruskalUnionFindCompressed.  Il grafico in figura
\ref{fig:Kruskal-KruskalCompressed-80k-100k} mostra una lievissima
differenza tra i due, che vede vincitore il primo. La differenza però
è veramente molto sottile ed entrambi gli algoritmi presentano una
piccola oscillazione che copre la differenza tra i tempi di esecuzione
tra i due. Le tecniche di ottimizzazione per union find quindi, nel
nostro caso, non sono state efficaci.

\begin{figure}[H]
    \centering
    \includegraphics[width=1.0\textwidth]{./images/KruskalUnionFind_vs_KruskalUnionFindCompressed_(nodes_between_80k_and_100k).png}
	\caption{Andamento di KruskalUnionFind e KruskalUnionFindCompressed con taglia dell'input da 80k a 100k nodi.}
    \label{fig:Kruskal-KruskalCompressed-80k-100k}
\end{figure}

% path-compression via path-halving + union by size
