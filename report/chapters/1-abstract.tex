\section{Abstract}
\label{cap:abstract}

Questo primo homework di laboratorio di Algoritmi Avanzati ha lo scopo di implementare e confrontare algoritmi per il calcolo del Minimum Spanning Tree di grafi connessi non diretti e pesati. \\

\noindent Gli algoritmi implementati sono tre:

\begin{enumerate}
    \item Algoritmo di Prim implementato con la struttura dati MinHeap;
    \item Algoritmo di Kruskal nella sua implementazione ``naive'' di complessità $\bigO{m \cdot n}$;
    \item Algoritmo di Kruskal implementato con la struttura dati Disjoint-Set (Union-Find).
\end{enumerate}

\noindent Abbiamo considerato anche alcune estensioni originali rispetto agli algoritmi visti in classe; esse sono discusse e presentate nella sezione \hyperref[cap:extensions-and-originalities]{Estensioni e originalità}. \\

\noindent Il codice è scritto in C++17 ed è opportunamente commentato per facilitarne la comprensione. Non è stata usata alcuna libreria esterna. Le uniche strutture dati impiegate offerte dalla libreria standard del linguaggio sono:

\begin{itemize}
    \item \textbf{std::vector}: array dinamici che permettono accesso casuale in tempo costante;
    \item \textbf{std::unordered\_map}: container associativo di tipo \textit{Hash Table};
    \item \textbf{std::unordered\_set}: container associativo che contiene un insieme di oggetti univoci;
    \item \textbf{std::pair}: tupla di due elementi.
\end{itemize}
Le altre strutture dati necessarie all'implementazione degli algoritmi MST sono state definite \textit{``from scratch''}.

\noindent Le risposte alle 2 domande principali dell'homework sono riportate nella sezione \hyperref[cap:performance-analysis]{Analisi delle performance}.
