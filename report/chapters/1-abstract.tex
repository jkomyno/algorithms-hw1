\section{Abstract}
\label{cap:abstract}

Questo primo homework di laboratorio di Algoritmi Avanzati ha lo scopo di implementare e confrontare algoritmi per il calcolo del Minimum Spanning Tree di un grafo.
Sono stati considerati solo grafi non diretti pesati. \\

\noindent Gli algoritmi implementati sono 3:

\begin{enumerate}
    \item Algoritmo di Prim implementato con la struttura dati MinHeap;
    \item Algoritmo di Kruskal nella sua implementazione "naive" di complessità O(mn);
    \item Algoritmo di Kruskal implementato con la struttura dati Disjoint-Set (Union-Find).
\end{enumerate}

\noindent Abbiamo considerato anche alcune estensioni originali rispetto agli algoritmi visti in classe; esse sono discusse e presentate nella sezione \hyperref[cap:extensions-and-originalities]{Estensioni e originalità}. \\

\noindent Il codice è scritto in C++17 ed è opportunamente commentato per facilitarne la comprensione. Non è stata usata alcuna libreria esterna. \\

\noindent Le risposte alle 2 domande principali dell'homework sono riportate nella sezione \hyperref[cap:performance-analysis]{Analisi delle performance}.
