\section{Scelte implementative}
\label{cap:implementation-choices}

\subsection{Rappresentazione del grafo}

Le due possibilità standard per rappresentare un grafo sono \textbf{Matrici di adiacenza} e \textbf{Liste di adiacenza}.

\subsection{Codice templatizzato}

\subsubsection{Strutture Dati}

\subsubsection{Binary Heap}

\iffalse
The class that implements a Binary Heap data structure
can be instantiated as a MinHeap or a MaxHeap at compile-time. \\

We spared the \complexityN{} time required to build the heap at the beginning of Prim's algorithm because the \textit{std::vector} we use to initialize the list already respects the MinHeap property: the key related to the first element is 0, and every other key is $+\infty$. \\
\fi

\subsubsection{Priority Queue}

\subsubsection{Disjoint Set}

\subsection{Algoritmi}

\subsubsection{Prim con Binary Heap}

\subsubsection{Kruskal Naive}

\subsubsection{Kruskal con Disjoint Set}
