\section{Linguaggio di programmazione scelto}
\label{cap:language-choice}

Abbiamo scelto di usare C++17 per implementare questo homework. Le ragioni sono principalmente dovute al fatto che:

\begin{itemize}
    \item è un linguaggio compilato e fortemente orientato ai tipi;
    \item è privo di garbage collector, il che velocizza l'esecuzione del codice;
    \item supporta i tipi generici (\textit{template});
    \item permette di creare personalizzazioni a compile-time senza alcun impatto a runtime;
    \item ha un'ampia libreria standard consolidata e ben mantenuta;
    \item se opportunamente usato, permette di limitare al minimo il consumo di memoria;
    \item la sua sintassi è intuitiva, leggibile e facile da seguire, anche se un po' più prolissa rispetto ad altri linguaggi;
    \item supporta la \textit{move semantics} e altre ottimizzazioni che permettono al programmatore di evitare inutili copie di oggetti e overhead in memoria.
\end{itemize}

\subsection{IDE e compilatore}

Poiché il nostro sistema operativo di sviluppo è Windows 10, abbiamo usato l'IDE Visual Studio 2019 Community e il suo compilatore \textit{MSVC v142 x64/x86}. \\

\noindent Nell'archivio allegato a questa relazione abbiamo incluso un \textit{Makefile} per permettere la compilazione su altri sistemi operativi usando \textit{g++9.3}.
